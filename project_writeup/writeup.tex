\documentclass[12pt]{article}

\usepackage[margin=1in]{geometry}
\usepackage{amssymb}
\usepackage{listings}
\usepackage{xcolor}
\usepackage{graphicx}
\usepackage{float}
\usepackage[hidelinks]{hyperref}
\usepackage{pgfplots}
\usepackage{enumitem}
\usepackage{bm}
\usepackage{amsmath}
\usepackage{amsthm}
\usepackage{titling}
\newtheorem{theorem}{Theorem}
\newtheorem{definition}{Definition}
\newcommand{\subtitle}[1]{\posttitle{\par\end{center}\begin{center}\large#1\end{center}\vskip0.5em}}
\definecolor{codegreen}{rgb}{0,0.6,0}
\definecolor{codegray}{rgb}{0.4,0.4,0.4}
\definecolor{codepurple}{rgb}{0.58,0,0.82}
\definecolor{backcolour}{rgb}{0.91,0.91,0.91}
\lstdefinestyle{mystyle}{backgroundcolor=\color{backcolour}, commentstyle=\color{codegreen}, keywordstyle=\color{magenta}, numberstyle=\footnotesize\color{codegray}, stringstyle=\color{codepurple}, basicstyle=\ttfamily\fontsize{12}{12}, breakatwhitespace=false, breaklines=true, captionpos=b, keepspaces=true, numbers=left, numbersep=5pt, showspaces=false, showstringspaces=false, showtabs=false, tabsize=2}
\lstset{style=mystyle}

\newcommand{\code}[1]{\colorbox{backcolour}{\texttt{#1}}}

\begin{document}

\title{CTA200H Final Project Writeup}
\author{Anatoly Zavyalov}
\date{\today}
\maketitle

\section{Abstract}

\section{Installation \& Setup}

The following steps outline how to install JanusGraph along with the Apache Cassandra storage backend\footnote{https://cassandra.apache.org/} and the Elasticsearch indexing backend\footnote{https://www.elastic.co/elasticsearch/}, as well as setting up gremlinpython, which is used for querying the JanusGraph backend from a Python interface. The operating system used is \textbf{Ubuntu 20.04.2 LTS}.

\subsection{Installing Java}

JanusGraph is built on top of Apache Tinkerpop \footnote{https://tinkerpop.apache.org/}, which, in turn, is built on top of Java and hence requires Java SE 8. The implementation of Java that we will install is OpenJDK 1.8. First, we refresh the list of available packages:
\begin{lstlisting}[numbers=none]
$ sudo apt update
\end{lstlisting} 

Next, we install OpenJDK 1.8:
\begin{lstlisting}[numbers=none]
$ sudo apt install openjdk-8-jre
\end{lstlisting}

To verify that the correct version has been installed, we run \code{java -version}. We should see something similar to \code{openjdk version "1.8.0\_292"}. 

\subsection{Setting the \code{\$JAVA\_HOME} environment variable}
Next, we must set the \code{\$JAVA\_HOME} environment variable. First, we head to \code{/usr/lib/jvm/} and locate the intallation of the JDK. It should look similar to \code{/usr/lib/jvm/java-11-openjdk-amd64}. Next, we set the \code{\$JAVA\_HOME} environment variable to point to the installation of the JDK:
\begin{lstlisting}[numbers=none]
$ export JAVA_HOME=/usr/lib/jvm/java-11-openjdk-amd64 
\end{lstlisting}

We doublecheck that this is successful with \code{echo \$JAVA\_HOME}.

\subsection{Setting up JanusGraph}

From the JanusGraph Releases\footnote{https://github.com/JanusGraph/janusgraph/releases}, we download the .zip of the ``full'' installation of JanusGraph (the file name should resemble \code{janusgraph-full-X.X.X.zip}, where \code{X.X.X} is the version number), and extract the contents. This ``full'' installation includes JanusGraph, as well as pre-configured Apache Cassandra and Elasticsearch.

From here, we start the JanusGraph server by running
\begin{lstlisting}[numbers=none]
$ bin/janusgraph.sh start
\end{lstlisting}

We can then open the Gremlin console by running
\begin{lstlisting}[numbers=none]
$ bin/gremlin.sh
\end{lstlisting}

Next, we may create a remote connection to the JanusGraph server:
\begin{lstlisting}[numbers=none]
gremlin> :remote connect tinkerpop.server conf/remote.yaml
\end{lstlisting}

From here, we can send commands to the JanusGraph server by preceding them with \code{:>}. We can avoid this by running 
\begin{lstlisting}[numbers=none]
gremlin> :remote console
\end{lstlisting}
which will enable sending all queries directly to the JanusGraph server and avoid the need of \code{:>}.



\section{Timestamp System}

\section{Scaling}

\section{Further Discussion}


\end{document}